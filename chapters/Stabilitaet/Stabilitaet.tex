% !TeX spellcheck = de_CH_frami
\section{Stabilität von MOS Operationsverstärker (Kap. 13)}
Loop-Gain $T(s)$: $T(s) = A(s)\cdot F(s)$\\
\begin{tabular}{|l|l|}
	\hline
	Phasenmarge bei& Verhalten des Verstärkers\\
	$f_{krit}$ ($a_L = 1$)& (System mit zwei weit auseinanderliegenden Polen)\\ \hline
	$\varphi_M \leq \SI{0}{\degree}$& Gegenkoppelter Verstärker schwingt selbständig\\ \hline
	$\varphi_M > \SI{0}{\degree}$& Gedämpftes Überschwingen der Sprungantwort\\ \hline
	$\varphi_M = \SI{65}{\degree}$& Peaking verschwindet. Einziger Überschwinger mit \SI{4.7}{\percent} Sprunghöhe\\ \hline
	$\varphi_M \geq \SI{75}{\degree}$& Kein Überschwingen\\ \hline
\end{tabular}\\[2ex]
\begin{tabular}{ll}
	Stabilitätskriterien&$\phi = \SI{180}{\degree} => |A(s)\cdot F(s)| < 1$\\
	&$|A(s)\cdot F(s)| = 1 => \SI{180}{\degree}-\Phi > 0; \varphi_M > \SI{0}{\degree}$\\
	Phasenmarge&$\varphi_M = \SI{180}{\degree}-\Phi = \SI{90}{\degree}-arctan(\frac{GBP}{f_{P2}})$\\
	Designregel&Wähle 2. Pol ($f_{nd}$) bei ca. $3\cdot GBP$\\
	&Dies ergibt eine Phasenmarge von \SI{72}{\degree} und somit kaum Überschwingen
\end{tabular}\\[1ex]
Jeder Knoten $N$ bildet einen Pol bei der Frequenz $f_N$, der sich wie folgt berechnet: b$f_N=\frac{1}{2\pi\cdot R_N C_N}$\\
\textbf{Grobe Analyse:} Die Knoten mit hohen RC-Produkten suchen. Dort entstehen Systempole, welche einen Abfall von \SI{20}{\decibel/Dekade} im Frequenzgang einleiten.
