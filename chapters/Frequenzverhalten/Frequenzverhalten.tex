% !TeX spellcheck = de_CH_frami

\section{Frequenzverhalten von MOS-Verstärkern (Kap. 11)}
\begin{minipage}[c]{0.5\textwidth}
	\includegraphics[width=1\linewidth]{chapters/Frequenzverhalten/images/parasitaere_kapazitaeten}
\end{minipage}
\begin{minipage}[c]{0.5\textwidth}
\textbf{Analyse mittels Open-Circuit Time Constant Methode} \\
Bestimmung des für die Bandbreite verantwortlichen Pols:
\begin{compactenum}
	\item Setze alle unabhängigen Quellen = 0 (V $\rightarrow$ Kurzschluss, I $\rightarrow$ Leerlauf)
	\item Berechne für alle Kapazitäten die zugehörige Zeitkonstante, wenn alle anderen C = 0.
\end{compactenum}
\end{minipage}
\begin{compactenum}
	\setcounter{enumi}{2}
	\item Bestimme die Bandbreite als Summe der Zeitkonstanten: $\omega_{-3dB}\approx \frac{1}{\sum\tau_k}=\frac{1}{\sum R_kC_k}$
\end{compactenum}
Frequenz des dominierenden Pols $f_d$ und somit die Bandbreite: $f_d=\frac{1}{2\pi \cdot R_dC_d}$

