% !TeX spellcheck = de_CH_frami

\section{Frequenzverhalten von MOS-Verstärkern (Kap. 11)}
\subsection{Analyse mittels Open-Circuit Time Constant Methode} 
\begin{minipage}[c]{0.5\textwidth}
	\includegraphics[width=1\linewidth]{chapters/Frequenzverhalten/images/parasitaere_kapazitaeten}
\end{minipage}
\begin{minipage}[c]{0.5\textwidth}
\textbf{Bestimmung des für die Bandbreite verantwortlichen Pols:}
\begin{compactenum}
	\item Setze alle unabhängigen Quellen = 0 (V $\rightarrow$ Kurzschluss, I $\rightarrow$ Leerlauf)
	\item Berechne für alle Kapazitäten die zugehörige Zeitkonstante, wenn alle anderen C = 0.
\end{compactenum}
\end{minipage}
\begin{compactenum}
	\setcounter{enumi}{2}
	\item Bestimme die Bandbreite als Summe der Zeitkonstanten: $\omega_{-3dB}\approx \frac{1}{\sum\tau_k}=\frac{1}{\sum R_kC_k}$
\end{compactenum}
Frequenz des dominierenden Pols $f_d$ und somit die Bandbreite: $f_d=\frac{1}{2\pi \cdot R_dC_d}$\\
\textbf{Vorgehen für Frequenzanalyse:}
\begin{compactenum}
	\item DC Verstärkung berechnen (aus Kleinsignalersatzschaltung für tiefe Frequenzen)
	\item Die für die Übertragungsfunktion relevanten Pole (und Nullstellen) des	Systems finden. Frage: An welchem Knoten befinden sich gleichzeitig ein hoher Widerstand und eine grosse Kapazität, also ein grosses RC-Produkt.
	\item Einzeichnen der einzelnen Pole (und allenfalls Nullstellen) ins Bode-Diagramm.
\end{compactenum}
Für eine grobe Analyse des Frequenzverhaltens genügt es somit, die Knoten zu suchen, bei denen das grösste RC-Produkte auftritt. Dort wird der dominierende Pol entstehen, welcher einen Abfallen des Frequenzgangs um 20dB/Dekade einleiten.
\subsection{Widerstände}
\begin{tabular}{lll}
	Knotenimpedanz praktisch unendlich: &$r_{iG}\rightarrow \infty$ & Gate \\
	Knotenimpedanz sehr hoch: &$r_{ds} = \frac{1}{g_0}$ & Drain wenn Stromquelle\\
	Knotenimpedanz tief: &$\frac{1}{g_m}$ & Drain wenn Diodenschaltung, Source wenn Stromquellenschaltung \\
\end{tabular}
\subsection{Kapazitäten}
\textbf{Wirksamkeiten der Kapazitäten:}
\begin{compactenum}
	\item Knotenkapazität gross: Knoten mit C als passive Schaltungskomponente
	\item Knotenkapazität mittel: parasitäre Kapazität verstärkt durch Miller-Effekt. Häufig $C_{GD}$ eines verstärkenden Transistors.
	\item Knotenkapazität klein: Knoten mit parasitären Kapazitäten. Von diesen Knoten ist in der Regel der Gate-Knoten mit der höchsten Kapazität belastet.
\end{compactenum}
\begin{tabular}{|l|l|l|l|l|}
	\hline
	{Arbeitsbereich} & \textbf{$C_{GS}$} & \textbf{$C_{GD}$} & \textbf{$C_{SB}$}& \textbf{$C_{DB}$} \\
	\hline
	Gesättigt & $C_{GS0}+2/3C_{oxt}$ & $C_{GD0}$ & $C_{jSBt}+2/3C_{BCt}$ & $C_{jDBt}$ \\
	Typ. Wert & 33\si{\femto \farad} & 1.2\si{\femto \farad} & 10\si{\femto \farad} & 7\si{\femto \farad} \\
	\hline
	Ungesättigt & $C_{GS0}+1/2C_{oxt}$ & $C_{GD0}+1/2C_{oxt}$ & $C_{jSBt}+1/2C_{BCt}$ & $C_{jDBt}+1/2C_{BCt}$ \\
	Typ. Wert & 26\si{\femto \farad} & 26\si{\femto \farad} & 10\si{\femto \farad} & 10\si{\femto \farad} \\
	\hline
\end{tabular} \\ [1ex]
$C_{oxt}=C_{ox}\cdot W \cdot L_{eff}$ \hspace{0.5mm} $C_{BCt}=C_{jBC}\cdot W \cdot L_{eff}$\\
$C_{jSBt} = C_{jSB} \cdot A_S + C_{jswSB} \cdot P_S$\\
$C_{jDBt} = C_{jDB}\cdot A_D + C_{jswDB}\cdot P_D$\\

Wenn $V_{SB}=0$, dann $C_{SB}$ ignorieren und $C_{DB}=C_{DS}$.
\subsection{Miller Effekt}
\begin{minipage}[c]{0.5\textwidth}
	\includegraphics[width=1\linewidth]{chapters/Frequenzverhalten/images/miller}
\end{minipage}
\begin{minipage}[c]{0.5\textwidth}
	Die Miller-Kapazität Cm , die zwischen Ein- und Ausgang eines Verstärkers mit Verstärkung A liegt, erscheint
	\begin{compactitem}
		\item multipliziert mit (1-A) parallel zum Eingang ($C_{mi}$ : Miller Eingangs-Kapazität)
		\item multipliziert mit (1-1/A) parallel zum Ausgang ($C_{mo}$ : Miller Ausgangskapazität)
		\item $_Cm$ wird aus dem Schema entfernt und durch $C_{mi}$ und $C_{mo}$ ersetzt.
	\end{compactitem}
\end{minipage}
